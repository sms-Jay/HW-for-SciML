\documentclass[11pt,a4 paper,one side]{article}
\usepackage{amsmath,amssymb,graphicx,subcaption}
\usepackage{ctex}  
\usepackage[colorlinks=true,linkcolor=red,citecolor=red,filecolor=magenta,urlcolor=cyan]{hyperref}
\usepackage{bookmark}
\usepackage{fontspec}
\setmainfont{Times New Roman}
\usepackage{xcolor}
\usepackage{geometry}
\geometry{a4paper, left=2.5cm, right=2.5cm, top=2.5cm, bottom=2.5cm}
\title{科学机器学习+HW2报告}
\author{2100012131 蒋鹏}
\date{\today}
\begin{document}
\maketitle
\tableofcontents
\section{问题描述}
考虑一维高斯过程$f(x) ~\sim GP(m(x),\kappa(x,x'))$,其中期望函数$m(x)=0$,核函数为\begin{equation}
    \kappa(x,x')=\sigma_f^2\exp(-\frac{(x-x')^2}{2l^2})+\sigma^2\delta_{x,x'}
\end{equation}
随机地在$(-8,8)$采样20个$x_i$,即$x_i\sim U(-8,8)$。用$(l,\sigma_f,\sigma)=(1.0,1.0,0.1)$对应的高斯过程采样相应的点$f(x_i)$。把$(x_i,f(x_i))$作为数据保存下来。
注意我们的代码指定了随机种子便于分析,将对应代码注释掉可以得到任意随机数。
\section{问题1}
使用$(l,\sigma_f,\sigma)=(1.0,1.0,0.1),(l,\sigma_f,\sigma)=(0.3,1.08,5e-5),(l,\sigma_f,\sigma)=(3.0,1.16,0.89)$的高斯过程进行回归,画出$(-8,8)$上的期望和不确定性。
\subsection{预期结果}
第一组参数为真实参数,拟合效果应该最好;第二组参数长度尺度很小,噪声极小,会导致回归曲线剧烈震荡,产生过拟合;第三组参数长度尺度很大,噪声较大,回归曲线平滑,
不确定性很大,欠拟合。
\subsection{数值方法}
用Cholesky分解求解某些量,然后根据理论结果进行均值与方差的预测。
\subsection{数值结果与分析}
我们展示拟合结果分别如图\ref{高斯过程回归 - 参数集1}、\ref{高斯过程回归 - 参数集2}、\ref{高斯过程回归 - 参数集3}所示,可以看到与预期结果相一致。
\begin{figure}[htbp]
    \centering
    \includegraphics[width=\linewidth]{高斯过程回归 - 参数集1.png}
    \caption{高斯过程回归 - 参数集1}
    \label{高斯过程回归 - 参数集1}
\end{figure}
\begin{figure}[htbp]
    \centering
    \includegraphics[width=\linewidth]{高斯过程回归 - 参数集2.png}
    \caption{高斯过程回归 - 参数集2}
    \label{高斯过程回归 - 参数集2}
\end{figure}
\begin{figure}[htbp]
    \centering
    \includegraphics[width=\linewidth]{高斯过程回归 - 参数集3.png}
    \caption{高斯过程回归 - 参数集3}
    \label{高斯过程回归 - 参数集3}
\end{figure}
\section{问题2}
假设我们已知核函数的形式为平方指数核,使用数据,采用贝叶斯模型选择方法,计算超参数的对数概率$\log \rho(\theta=(l,\sigma_f,\sigma)|y,X)$,优化求解出最好的超参。
画出对数概率的等高线图检验结果是否合理。用求出来的超参对应的高斯过程对数据进行回归,画出$(-8,8)$上的期望和不确定性。
\subsection{数值方法}
根据理论结果,该对数概率具有表达式,用L-BFGS-B结合多起点优化进行求解超参数,选择不同优化起点个数为100。
\subsection{数值结果与分析}
平方指数核最优超参数:$(l,\sigma_f,\sigma)=(1.2575,0.9936,0.1282)$。
首先展示对数概率分别在固定一个参数下的等高线图,如图\ref{平方指数核对数边际似然等高线图}所示。
\begin{figure}[htbp]
    \centering
    \begin{subfigure}{0.30\textwidth}
        \includegraphics[width=\textwidth]{平方指数核对数边际似然等高线图 (固定l=1.2575).png}
        \caption{平方指数核对数边际似然等高线图 (固定l)}
        \label{平方指数核对数边际似然等高线图 (固定l)}
    \end{subfigure}
    \begin{subfigure}{0.30\textwidth}
        \includegraphics[width=\textwidth]{平方指数核对数边际似然等高线图 (固定σ_f=0.9936).png}
        \caption{平方指数核对数边际似然等高线图 (固定$\sigma_f$)}
        \label{平方指数核对数边际似然等高线图 (固定σ_f)}
    \end{subfigure}
    \begin{subfigure}{0.30\textwidth}
        \includegraphics[width=\textwidth]{平方指数核对数边际似然等高线图 (固定σ_n=0.1282).png}
        \caption{平方指数核对数边际似然等高线图 (固定$\sigma_n$)}
        \label{平方指数核对数边际似然等高线图 (固定σ_n)}
    \end{subfigure}
    
    \caption{平方指数核对数边际似然等高线图}
    \label{平方指数核对数边际似然等高线图}
\end{figure}
\par 从图中可以看出:所得超参数为全局最优,并未陷入局部极大。
\par 最后展示在最优超参下的高斯回归,如图\ref{贝叶斯模型选择平方指数核}所示。可以看到回归效果与真实参数\ref{高斯过程回归 - 参数集1}相接近,可以认为超参预测效果好。
\begin{figure}[htbp]
    \centering
    \includegraphics[width=\linewidth]{贝叶斯模型选择 - 平方指数核.png}
    \caption{贝叶斯模型选择平方指数核}
    \label{贝叶斯模型选择平方指数核}
\end{figure}
\section{问题3}
考虑Matern类核函数\begin{equation}
    \kappa(x,x')=\sigma_f^2(1+\frac{\sqrt{3}}{l}|x-x'|)\exp(-\frac{\sqrt{3}}{l})+\sigma^2\delta_{x,x'}
\end{equation}假设我们已知核函数的形式为Matern核,使用数据,采用贝叶斯模型选择方法,计算超参数的对数概率$\log \rho(\theta=(l,\sigma_f,\sigma)|y,X)$,优化求解出最好的超参。
画出对数概率的等高线图检验结果是否合理。用求出来的超参对应的高斯过程对数据进行回归,画出$(-8,8)$上的期望和不确定性。
\subsection{数值方法}
根据理论结果,该对数概率具有表达式,用L-BFGS-B结合多起点优化进行求解超参数,选择不同优化起点个数为100。
\subsection{数值结果与分析}
平方指数核最优超参数:$(l,\sigma_f,\sigma)=(1.7236,0.9834,0.0951)$。
首先展示对数概率分别在固定一个参数下的等高线图,如图\ref{Matern核对数边际似然等高线图}所示。
\begin{figure}[htbp]
    \centering
    \begin{subfigure}{0.30\textwidth}
        \includegraphics[width=\textwidth]{Matern核对数边际似然等高线图 (固定l=1.7236).png}
        \caption{Matern核对数边际似然等高线图 (固定l)}
        \label{Matern核对数边际似然等高线图 (固定l)}
    \end{subfigure}
    \begin{subfigure}{0.30\textwidth}
        \includegraphics[width=\textwidth]{Matern核对数边际似然等高线图 (固定σ_f=0.9834).png}
        \caption{Matern核对数边际似然等高线图 (固定$\sigma_f$)}
        \label{Matern核对数边际似然等高线图 (固定σ_f)}
    \end{subfigure}
    \begin{subfigure}{0.30\textwidth}
        \includegraphics[width=\textwidth]{Matern核对数边际似然等高线图 (固定σ_n=0.0951).png}
        \caption{Matern核对数边际似然等高线图 (固定$\sigma_n$)}
        \label{Matern核对数边际似然等高线图 (固定σ_n)}
    \end{subfigure}
    
    \caption{Matern核对数边际似然等高线图}
    \label{Matern核对数边际似然等高线图}
\end{figure}
\par 从图中可以看出:所得超参数为全局最优,并未陷入局部极大。
\par 最后展示在最优超参下的高斯回归,如图\ref{贝叶斯模型选择Matern核}所示。可以看到回归效果与真实参数\ref{高斯过程回归 - 参数集1}相接近,可以认为超参预测效果好。
\begin{figure}[htbp]
    \centering
    \includegraphics[width=\linewidth]{贝叶斯模型选择 - Matern核.png}
    \caption{贝叶斯模型选择Matern核}
    \label{贝叶斯模型选择Matern核}
\end{figure}
\end{document}